\setchapterpreamble[u]%{\dictum[Larry Wall]{I don't know\\if it's what you want,\\but it's what you get. :-)}}

\chapter{Realization} \label{chap:Realization}
\minitoc\vspace{1em}

The problem is realized as a two part solution using the Client-Server architecture.


\section{Background}

\subsubsection{Java Technology}
The implementation of this project is done using Java\cite{??}.
Java technology refers to both the programming language and the platform.
The Java programming language is high level object oriented language\cite{Gosling:2014:JLS:2636997}, 
which is actively developed and supported. 
The Java Software Platform provides a system for developing application software and deploying it in a cross-platform computing environment. 
There are several Java platforms.
The Enterprise Edition (Java EE) platform provides an API and runtime environment for developing and running enterprise software, such as web services.

\subsubsection{JGraphT}

JGraphT is an open-source Java library specialized for graph-theory modelling and algorithms.It suppoerts several different types of graphs\cite{JGraphT}:
\begin{itemize}
	\item[--] Directed and undirected graphs
	\item[--] Graphs with labeled, unlabeled, weighted or unweighted edges
	\item[--] Edge multiplicity (more than one edge between two vertices)
	\item[--] Composition of all above
\end{itemize}

The JGraphT API implements vast number of important algorithms, such as PageRank, Random Walk, graph analysis, centrality etc.
JGraphT also provides an integration with the JGraph\footnote{\url{https://jgraph.com/}} library for elementary graph visualization.

\subsubsection{Gephi}

Gephi - The Open Graph Viz Platform is a open-source graph analysis tool, 
which provides rich visualizations and advanced exploration\cite{Gephi}.
The graphs can be of directed, undirected or mixed type\cite{Gephi:GraphAPI}.
Gephi offers filtering of the graphs by using attributes or built-in properties, 
such as in-degree to get more compact representation of the graph. 
It also supports determining modularity, centrality, detecting communities and similar layering algorithms.
Gephi is realized as a third-party tools, intended for non specialists i.e\. easy to use.

\subsubsection{R}

R is a language and environment for statistical computing and graphics\cite{R-project}.
It provides support for machine learning techniques and data analysis (classification, regression, time series analysis etc.).
The extension of the functionalities of R is done through packages.
The main repository Cran\footnote{\url{https://cran.r-project.org/web/packages}} 
contains 9833 packages as the time of writing (January 2017).

\section{Data Analysis Stages} \label{sec:steps}
According to \cite{Ojeda:2014:PDS:2721420}, we divided the analysis in five stages:
\begin{enumerate}
	\item \textbf{Data acquisition and importing}: 
	The first step in the pipeline is to acquiring the data and import it into our system. 
	The data can come from different sources and can exist in various formats.
	\item \textbf{Data wrangling and manipulation}: 
	Data is rarely in the desired format. 
	This step includes transforming and bringing the data in more convenient format.
	The purpose is to allow easy consumption for analysis.
	\item \textbf{Exploratory Analysis}: 
	The third step is to come to an understanding of the data that will be used. 
	This includes steps like visualization, summarization of characteristics, counting etc.
	\item \textbf{Modeling and analysis}: 
	This step is the core of the work. 
	Here we build and test various models for similarity, clustering and other machine learning use-cases.
	\item \textbf{Integrating and operationalizing}: 
	At the end of the pipeline we need to make use of the data back in a compelling form and structure,
	both for ourselves and as a response for API requests. 
\end{enumerate}

\section{Server Side}
The steps listed in \ref{sec:steps} are implemented on the server side.

\section{Client Side}